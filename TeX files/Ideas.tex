\section*{Ideas}

Flashee una estructura (obvio cambiala a gusto y piacere) y para cada grafico o mapa puse extractos del paper que hacen referencia. 
\\

\textbf{Primera parte:} graficos y mapas sobre el estado de situacion de Mexico. Osea cosas previas a los resultados que ayudan a reflejar como era la situación del mercado laboral (informalidad, asegurados, etc.) en México. 

\begin{itemize}
    \item Histograma de informalidad (data del censo año 2000): \\"The Mexican labor market is archetypical of a middle-income country. A large share of the labor force (over 50 percent) is classified as informal. In our case, the division between formal and informal labor is, nevertheless, very clear: formal workers are those contributing to IMSS and informal workers are not." \\
    \\También puede agregarse un mapa con la tasa de informalidad por municipio. O hacer un group by estado y hacer mapa con divisiones estatales. 
    \item Histograma de insured (data del censo año 2000):\\" Nonetheless, by 2000, the inequalities in this system were apparent: nearly 50 percent of the Mexican population, amounting to 47 million people, was not insured through either IMSS or ISSSTE and were relying on the SSA or private institutions for their health care"\\
   \\ También puede agregarse un mapa con la tasa de insured por municipio.  O hacer un group by estado y hacer mapa con divisiones estatales. 
    \item Mapa de municipios rurales y urbanos 
\end{itemize}

\textbf{Segunda parte:} implementación de SP 
\begin{itemize}
  \item GIF (o set de mapas) de como los estados van implementando el SP a lo largo de 2002-2007. \\"We exploit the variation generated by the time-staggered entry of municipalities into the program. The program started as a pilot during 2002 in five states and by the end of 2007 virtually all municipalities in the country had enrolled in the program.  The SP was implemented in stages across states. Passed into law in 2004 as a modification of the existing General Health Law, the program actually began with a pilot phase in five states in 2002 (Colima, Jalisco, Aguascalientes, Tabasco, and Campeche). During 2002 and 2003, 14 other states15 started to implement the SP with- out a formal agreement with the federal government"
  \item mapas (o grafico) de cómo va cambiando la cantidad de familias beneficiarias del SP a lo largo de los años.  Ó grafico de la proporcion de individuos por municipio afiliados al SP a lo largo del tiempo. dividir indiivudos afiliados (benef\_SP\_2002\_2009) por la poblacion del municip (Population\_poverty) : \\ "We merge this data with the administrative records of SP by municipality. In those records we observe the number of families and individuals affiliated to the SP in each quarter from 2002–2009. We define that the SP is operating in a municipality if the number of individuals affiliated is greater than ten". \\
  "In the initial years of the pro- gram, the number of beneficiaries was low. For example, between 2002 and 2004, around one-third of municipalities were enrolled in the program and the number of registered families was around 1.5 million families, representing roughly 6 percent of the families in Mexico.17 By 2008, over 7.5 million families and 23 million individuals were affiliated with SP, representing around 30 percent of the total number of families in Mexico. The program expanded rapidly in 2009 and 2010 covering close to 50 million individuals by 2011 replacing IMSS as the largest health care system in the country in terms of number of affiliates·"\\
  \item scatterplot para mostrar corr positva: "We find that systematically more populated municipalities and those in smaller states (only in the panel municipalities) joined the program at earlier stages" data del censo 2002
  \item scatterplot para mostar que no hay corr: "We find that systematically more populated municipalities and those in smaller states (only in the panel municipalities) joined the program at earlier stages" data del censo 2002 
 
\end{itemize}

\textbf{Tercera parte:} Resultados 

\begin{itemize}
  \item plot de perdida de employer registration in small and medium firmas (hasta 50 employees) como consecuencia del SP:\\
  "Column 1 of Table 2 suggests that within the first year after the implementation of the program, employer affiliation to IMSS falls by 0.7 percent and by the end of the fourth year the effect reaches 4.4 percent. Importantly, we find virtually zero coefficients for the years before the implementation of the program suggesting a causal interpretation of our results. Columns 2–6 show that the effects are concen- trated among small and medium firms. We find significant negative effects of the SP on the registration of firms up to 50 employees, especially in firms between 2–5 employees, where the fall, four years after implementation, reaches 4.9 percent. We group the employers in firms from 1 up to 50. Our estimates (shown in column 7) do not change. Within the fourth year formal employment registration falls 4.6 percent."
  \hl{como verga grafico esto}
 \item plot de perdida de employee registration in small and medium firmas (hasta 50 employees) como consecuencia del SP:\\
 "the registration of employees with IMSS by the end of the fourth year after the implementation of the program fell by 3.8, 5.1, 3.3, and 3.9 percent for firms of 1, 2–5, 6–50, and 51–250 employees. If we aggregate for firms up to 50 employees (column 7) there is a fall of 4 percent after four years of implementation. In all these cases we do not observe any pretreatment trends in any of our results."
   \hl{como verga grafico esto}
 \item algo comparando los beneficios con los costos: \\
beneficios:\\ "The reform had clear positive impacts. Aguilera, Miranda, and Velázquez (2012) estimated that by 2010, for each \$1 spent in the SP there was a saving of between US \$0.35 and \$0.17 in catastrophic health expenditure of uninsured households. Taken at face value, this estimation implies that the SP reduced catastrophic expenditure of between 0.07 and 0.14 percent of GDP."\\
costos de la realocación del labour: "This reallocation had measurable costs. On the one hand it generated a loss of rev- enue to the Mexican social security of at least 0.62 percent of their annual revenue, equivalent to 2 percent of the budget of the SP and 0.01 percent of gross domestic product (GDP). Additionally, the loss of 36,000 registered small and medium firms could have generated a loss of VAT of between 0.08 and 0.32 percent of GDP and a drop between 0.03 and 0.09 percent of GDP." \\
"These reallocation losses seem to be in the same order of magnitude as some of the gains associated with expansion of the SP"\\
(ni idea con este, toy flasheando maybe pero lo dejo por las dudas) 
  \hl{y los datos?}
 \item scatter plot tamaño de municipio (entiendo que usa poblacion para medir tamaño but im not sure porque despues dice que en las regresiones controla por población del municipio. No esta claro) contra el number of registered employers and employees of firms with up to 50 employees. Para ver que hay mayor efecto entre las municipalidades mas chicas: \\
 ". We divide our sample of municipalities into four equal groups according to municipality size in the 2000 census. This corresponds to municipality sizes of less than 10,000, between 10,000 and 22,000, between 22,000 and 50,000, and more than 50,000."
 \hl{Este grafico no tiene sentido, es obvio que a mas poblacion va a ser mayor, no dice nada}
 \item algo similiar al anterior pero con el status de rural/urban del municipio. mayor efecto en los rurales

 \end{itemize}